\documentclass[]{article}

%opening
\title{Design of a UAV Radar Using COTS Automotive RADAR Components}
\author{Brandon Jennings and William Horn}

\begin{document}

\maketitle

\begin{abstract}
This document captures the design of a UAV RADAR system built using automotive RADAR components.
\end{abstract}

\section{Introduction and Overview}
The purpose of this document is to provide a reference for the design and trade study of a UAV RADAR system built using automotive RADAR components. Calculations, notes, and general musings will be found below.

\section{Transmitter}

\subsection{Pre-Amp: TGA4706-FC-1}
The pre-amp under consideration has a typical output power of 12 dBm.

\subsection{Microstrip / Patch Antenna}

The patch antenna is notoriously narrow band (3\%). For a 77 GHz transmission signal, that means that the bandwidth for the FM should be within 2.5 GHz. This information taken from http://www.antenna-theory.com/antennas/patches/antenna.php. 

The ground plane should extend beyond the edges of the patch by at least 2 to 3 times the board thickness for proper operation (http://orbanmicrowave.com/wp-content/uploads/2014/12/Orban-Patch-Antennas-2009-rev.pdf). A good approximation for resonant length is:

\begin{equation}
L \approx 0.49 \lambda_{d} = 0.49 \frac{\lambda_{0}}{\sqrt{\epsilon_{r}}}
\end{equation}

\noindent  where: \newline
$ L $ = resonant length \newline
$ \lambda_{d} $ = wavelength in PC board \newline
$ \lambda_{0} $ = wavelength in free space \newline
$ \epsilon_{r} $ = dielectric constant of the printed circuit board (4 for FR4)

Connecting the feed line to the patch is a matter of impedance matching for the application. Somewhere along the resonant length in the middle of the patch is a point that yields $ 50 \Omega $ of impedance, that's probably where we want to set our feed line.

\subsubsection{Antenna Gain}
From "Antenna Patterns and Their Meaning.pdf":
The gain of and antenna (in any given direction) is defined as the ratio of the power gain in a given direction to the power gain of a reference antenna in the same direction. It is standard practice to use an isotropic radiator as the reference antenna in this definition. It is customary to use dBi (decibels relative to an isotropic radiator) for gain with respect to an isotropic ratiator. Gain is expressed in dBi using the following formula:

\begin{equation}
GdBi = 10 \log{ \frac{G_{numeric}}{G_{isotropic}} } = 10 \log{ G_{numeric} }
\end{equation}

Occasionally, a theoretical dipole is used as the reference, so the unit dBd (decibels relative to a dipole) is used to describe the gain w.r.t. a dipole.

\section{Receiver}


\section{Digital Signal Processing (DSP)}

\subsection{Digital Circuitry}
This section details the conceptual design for the digital circuitry that is needed coming out of the AD8285 chip. My first thoughts are to use a FIFO for each channel of the receiver that is to be sampled. These FIFOs will store the samples collected from each sweep of the FMCW radar. That means that the size of each FIFO is dependent on both the sample rate of the ADC as well as the sweep period continuous wave.

Assuming a maximum sample rate of 72 MSPS and a maximum sweep period of 0.001 seconds, we can expect to store a maximum of 72,000 12 bit values. The FIFOs will then need to be transferred to the host processor for post processing (FFT, cross correlation, etc...). It is my desire to send the data over USB 3.0 to the processing device. A computer capable of handling the post processing functions should have a USB 3.0 interface. This also allows room for growth for further iterations of the RADAR.

\subsection{Post Processing}

\section{1D RADAR Performance Analysis}

\subsection{Alpha Configuration}

\subsubsection{Transmission}
A single transmission line starts with an input into the voltage controlled oscillator (VCO). It's likely that a microcontroller unit (MCU) can be used to supply the linear ramp voltage using an on-board digital to analog (D/A) converter. To keep the frequency ramp linear over all operating conditions, it's likely that a feedback function be integrated into this MCU. This allows "smart-chirp" operations in all operating conditions. An 8-bit PIC 16F690 (retails for \$2.00) is more than likely adequate.

The VCO part number is an HMC385 (retails for \$14.69). It provides a frequency range of 2.25 GHz to 2.5 GHz.


\section{2D RADAR Performance Analysis}

\section{3D RADAR Performance Analysis}

\end{document}
